\documentclass[a4paper]{book}
\usepackage[times,inconsolata,hyper]{Rd}
\usepackage{makeidx}
\usepackage[utf8]{inputenc} % @SET ENCODING@
% \usepackage{graphicx} % @USE GRAPHICX@
\makeindex{}
\begin{document}
\chapter*{}
\begin{center}
{\textbf{\huge Package}}
\par\bigskip{\large \today}
\end{center}
\begin{description}
\raggedright{}
\inputencoding{utf8}
\item[Type]\AsIs{Package}
\item[Title]\AsIs{Nice Plots for Data Exploration}
\item[Version]\AsIs{0.1.2}
\item[Author]\AsIs{Zachary Hunter}
\item[Maintainer]\AsIs{Zachary Hunter }\email{zhunter7@gmail.com}\AsIs{}
\item[Description]\AsIs{More about what it does (maybe more than one line)
Use four spaces when indenting paragraphs within the Description.}
\item[License]\AsIs{MIT + file LICENSE}
\item[Encoding]\AsIs{UTF-8}
\item[LazyData]\AsIs{true}
\item[RoxygenNote]\AsIs{6.0.1}
\item[Depends]\AsIs{tidyverse,beeswarm,sm,RColorBrewer}
\end{description}
\Rdcontents{\R{} topics documented:}
\inputencoding{utf8}
\HeaderA{addNicePoints}{Add a datapoint overlay to a box or violin plot}{addNicePoints}
%
\begin{Description}\relax
This function prepares data based on settings from \code{\LinkA{niceBox}{niceBox}}, \code{\LinkA{niceDots}{niceDots}}, or \code{\LinkA{niceVio}{niceVio}}
and passes the data on to \code{\LinkA{drawPoints}{drawPoints}}.
\end{Description}
%
\begin{Usage}
\begin{verbatim}
addNicePoints(prepedData, by, filter = TRUE, sidePlot = F, subGroup = F,
  plotAt, pointHighlights = F, pointMethod = "jitter", pointShape = 16,
  pointSize = 1, width = 1, pointLaneWidth = 0.9,
  plotColors = formatPlotColors(list(1)), drawPoints = T, outliers = F,
  dataCols = 1)
\end{verbatim}
\end{Usage}
%
\begin{Arguments}
\begin{ldescription}
\item[\code{prepedData}] list; a list object returned by \code{\LinkA{prepCategoryWindow}{prepCategoryWindow}}

\item[\code{by}] factor or dataframe of factors; One or more factors that control how the data is grouped. The first column is the primary grouping factor and the second and thrid columns are used for sub-grouping and highlighting as needed.

\item[\code{filter}] logical vector; Used to further filter the data if necissary.

\item[\code{sidePlot}] logical; switches the axis to plot horizontally instead of vertically.

\item[\code{subGroup}] logical; Should the data be faceted into subgroups within the primary factor levels. Ignored if \code{by} is a \code{\LinkA{factor}{factor}}.

\item[\code{plotAt}] numeric; A vector of where to draw each set of points

\item[\code{pointHighlights}] logical; will use additional factors in \code{by} to highlight points in the dot plot.

\item[\code{pointMethod}] character; method to be used for ploting dots. Can be set to "jitter", "linear", "beeswarm" or "distribution".

\item[\code{pointShape}] positive integer; sets pty for plotting data points. Can be a vector to support additional graphical customization.

\item[\code{pointSize}] positive integer; sets the cex multiplier for point size.

\item[\code{width}] numeric; A multiplier that controls how wide the ploting elements will be. Setting \code{width=1.1} would result in plot elements being 10\% wider.

\item[\code{pointLaneWidth}] numeric; This controls how far data point dots can move along the categorical axis when plotting. Used for \code{pointMethod} options 'jitter', 'beeswarm', and 'distribution'.

\item[\code{plotColors}] list; a named list of vectors of colors that set the color options for all NicePlot functions. Names left unspecified will be added and set to default values automatically.

\item[\code{drawPoints}] logical; draws a dot plot overlay of the data for each box. Setting this to false causes just the outlier points to be ploted. Used in \code{\LinkA{niceBox}{niceBox}}.

\item[\code{outliers}] positive numeric; number of interquartile ranges (IQR) past the Q1 (25\%) and Q3 (75\%) cumulative distribution values. Outliers are often defined as \eqn{1.5 \times IQR}{} and extreme outliers are more than \eqn{3 \times IQR}{} away from the inner 50\% data range.

\item[\code{dataCols}] numeric; A number of representing the number of data columns to be plotted. These is a combination of the dimentions of \code{prepedData} and/or the number of primary and secondary grouping factors. Used to determine the maximum ploting width for the points.
\end{ldescription}
\end{Arguments}
%
\begin{Details}\relax
This funciton takes in cleaned data from \code{\LinkA{prepCategoryWindow}{prepCategoryWindow}} and reorganizes to to create a dot plot overlay for a graph.
This code is used by both \code{\LinkA{niceBox}{niceBox}} and \code{\LinkA{niceVio}{niceVio}} and has been moved to an independant funciton to make the code more compact and easier to maintain.
This code is also used to draw the outlier dots in a boxplot by setting \code{drawPoints = \LinkA{FALSE}{FALSE}}.
\end{Details}
%
\begin{SeeAlso}\relax
\code{\LinkA{drawPoints}{drawPoints}}, \code{\LinkA{niceBox}{niceBox}}, \code{\LinkA{niceVio}{niceVio}}, \code{\LinkA{niceDots}{niceDots}}, \code{\LinkA{beeswarm}{beeswarm}}, \code{\LinkA{jitter}{jitter}}, \code{\LinkA{drawPoints}{drawPoints}}
\end{SeeAlso}
%
\begin{Examples}
\begin{ExampleCode}
#Add a beeswarm plot overlay to a boxplot in the iris dataset:
data(iris)
data<-list(data=iris$Sepal.Length)
boxplot(iris$Sepal.Length~iris$Species)
addNicePoints(data,by=iris$Species,pointMethod="beeswarm",plotAt=1:3)

#Add an outlier point to a boxplot:
boxplot(iris$Sepal.Length~iris$Species, outline=FALSE)
addNicePoints(data,by=iris$Species,pointMethod="linear",plotAt=1:3,
    drawPoints=FALSE,outliers=1.5)

\end{ExampleCode}
\end{Examples}
\inputencoding{utf8}
\HeaderA{basicTheme}{Nice Plots Theme: Basic}{basicTheme}
\keyword{datasets}{basicTheme}
%
\begin{Description}\relax
This is the default theme for nicePlots
\end{Description}
%
\begin{Usage}
\begin{verbatim}
basicTheme
\end{verbatim}
\end{Usage}
%
\begin{Format}
An object of class \code{list} of length 42.
\end{Format}
%
\begin{Details}\relax
This default theme has uses transparent solid circles for point overlays with up to 8 colors.
Fill and line options as constant.
\end{Details}
\inputencoding{utf8}
\HeaderA{calcStats}{calculate preliminary statistical significance analysis}{calcStats}
%
\begin{Description}\relax
\code{calcStats} takes a numeric vector and a factor and runs a preliminary statistical analysis. Output is printed to the screen and the p-value is returned as a character string.
\end{Description}
%
\begin{Usage}
\begin{verbatim}
calcStats(x, by, type = c("Wilcox", "Tukey", "T.Test", "ANOVA"),
  verbose = FALSE)
\end{verbatim}
\end{Usage}
%
\begin{Arguments}
\begin{ldescription}
\item[\code{x}] numeric; numeric vector of data points to analyze.

\item[\code{by}] factor; factor describing the groups within \code{x} to test.

\item[\code{type}] character; determines which statistical test should be used. Accepted values are 'wilcox', 't.test', 'ttest', 'anova' and 'tukey'. Values not matching a valid input will produce a warning.

\item[\code{verbose}] logical; will print statistical output to the screen if set \code{\LinkA{TRUE}{TRUE}}. Calculations returned by the function either way.
\end{ldescription}
\end{Arguments}
%
\begin{Details}\relax
This is designed to be used in conjunction with data visualization plots to help with data exploration and should not be used for a robust statistical analysis. Normal distribution, variance and other data characteristics are not evaluated and there is no guarantee that the underling test assumptions are met. For two level factors \code{\LinkA{wilcox.test}{wilcox.test}} or \code{\LinkA{t.test}{t.test}} is recommended. If the factor has more than two levels then \code{\LinkA{pairwise.wilcox.test}{pairwise.wilcox.test}} and \code{\LinkA{pairwise.t.test}{pairwise.t.test}} are automatically selected. In this case \code{\LinkA{anova}{anova}} and the optional follow-up \code{\LinkA{TukeyHSD}{TukeyHSD}} can also be used. All output it printed to the console and for the two level tests and \code{\LinkA{anova}{anova}} the p-value is returned as a text string.
\end{Details}
%
\begin{Value}
a character string describing the test run and the p-value.
\end{Value}
%
\begin{SeeAlso}\relax
\code{\LinkA{wilcox.test}{wilcox.test}}, \code{\LinkA{pairwise.wilcox.test}{pairwise.wilcox.test}}, \code{\LinkA{t.test}{t.test}}, \code{\LinkA{pairwise.t.test}{pairwise.t.test}}, \code{\LinkA{anova}{anova}}, \code{\LinkA{TukeyHSD}{TukeyHSD}}
\end{SeeAlso}
%
\begin{Examples}
\begin{ExampleCode}
data(iris)
pv<-calcStats(iris$Petal.Length,by=iris$Species,type="anova")
boxplot(iris$Petal.Length~iris$Species,main="Petal Length by Species",sub=pv)

\end{ExampleCode}
\end{Examples}
\inputencoding{utf8}
\HeaderA{dataFlightCheck}{Check data formating for the NicePlots package}{dataFlightCheck}
%
\begin{Description}\relax
Formats and cleans data prior to setting up the plotting enviroment
\end{Description}
%
\begin{Usage}
\begin{verbatim}
dataFlightCheck(data, by, flipFacts, na.rm = FALSE)
\end{verbatim}
\end{Usage}
%
\begin{Arguments}
\begin{ldescription}
\item[\code{data}] vector or dataframe; data to be plotted

\item[\code{by}] factor or dataframe; factors to be used to format data

\item[\code{flipFacts}] logical; If a dataframe is used for plotting data input, this will covert the data to a vector with the dataframe colums trasfered a factor in the second column of the \code{by} input.

\item[\code{na.rm}] logical; Removes all data and factor rows were \code{NA} is presenst.
\end{ldescription}
\end{Arguments}
%
\begin{Details}\relax
This funciton makes sure the \code{data} input is a numeric vector or a data frame of numeric vectors.
It will also check to make sure \code{by} is a factor or a dataframe of factors. If requested, it will also remove missing data and rearrage numeric dataframe inputs.
\end{Details}
%
\begin{Value}
A named list with \code{d=data} and \code{b=by}.
\end{Value}
%
\begin{Examples}
\begin{ExampleCode}
todo<-1

\end{ExampleCode}
\end{Examples}
\inputencoding{utf8}
\HeaderA{drawBar}{drawBar}{drawBar}
%
\begin{Description}\relax
Add a barplot with options error pars to the active ploting enviroment
\end{Description}
%
\begin{Usage}
\begin{verbatim}
drawBar(x, plotColors, errorBars = FALSE, errorCap = "ball",
  errorLineType = 1, width = 0.5, sidePlot = FALSE, stacked = FALSE,
  capSize = 2, lineWidth = 1)
\end{verbatim}
\end{Usage}
%
\begin{Arguments}
\begin{ldescription}
\item[\code{x}] dataframe; number of interquartile ranges (IQR) past the Q1 (25\%) and Q3 (75\%) cumulative distribution values. Outliers are often defined as \eqn{1.5 \times IQR}{} and extreme outliers are more than \eqn{3 \times IQR}{} away from the inner 50\% data range.

\item[\code{plotColors}] list; a named list of vectors of colors that set the color options for all NicePlot functions. Names left unspecified will be added and set to default values automatically.

\item[\code{errorBars}] Logical; Should error bars be drawn. Defaults to true but is ignored if \code{stack=\LinkA{TRUE}{TRUE}}.

\item[\code{errorCap}] character; Determines the style for the ends of the error bars. Valid options are \code{ball}, \code{bar} or \code{none}.

\item[\code{errorLineType}] numeric; Sets \code{lty} line type for drawing the error bars.

\item[\code{width}] numeric; cex like scaling factor controlling the width of the bars.

\item[\code{sidePlot}] logical; Plots bar hight on the x axis if set to \code{\LinkA{TRUE}{TRUE}}.

\item[\code{stacked}] logical; draws a stacked barplot if set to \code{\LinkA{TRUE}{TRUE}}.

\item[\code{capSize}] numeric; cex like scaling value the controls the size of the caps on the error bars.

\item[\code{lineWidth}] numeric; Sets the \code{lwd} options for controling line plotting thickness for the bar plot.
\end{ldescription}
\end{Arguments}
%
\begin{Details}\relax
This function draws a series of bars based on a dataframe. The expected columns include \code{yt} (locaion top of the bar),
\code{yb} or bottom of the bar, \code{at} indicating where the bar should be drawn, \code{Group} which is a unique ID per row, \code{fact} which contains an optional stacking factor
\code{UpperError} for the top of the error bar and \code{LowerError} for the location of the bottom of the error bar. The construction of the dataframe is handled automatically from input data
by \code{\LinkA{niceBar}{niceBar}}.
\end{Details}
%
\begin{SeeAlso}\relax
\code{\LinkA{barplot}{barplot}}, \code{\LinkA{niceBar}{niceBar}}, \code{\LinkA{errorBars}{errorBars}}
\end{SeeAlso}
%
\begin{Examples}
\begin{ExampleCode}
data(iris)
data<-iris  %>% group_by(Species) %>%
    summarize(yt=mean(Sepal.Length),yb=0,UpperError=sd(Sepal.Length),
    LowerError=sd(Sepal.Length)) %>%
    ungroup() %>% select(yt,yb,UpperError,LowerError,Group=Species) %>%
    bind_cols(at=1:3,fact=1:3)
plot(type="n",xlim=c(0,4),ylim=c(0,max(iris$Sepal.Length)),-1,xaxt="n")
drawBar(data,plotColors=list())

\end{ExampleCode}
\end{Examples}
\inputencoding{utf8}
\HeaderA{drawBoxPlot}{draw a custom box and whisker plot}{drawBoxPlot}
%
\begin{Description}\relax
takes a date frame with columns labeled 'at', 'q1', 'q3', 'min', 'max', 'median' and 'width' to draw a series of boxplots.
\end{Description}
%
\begin{Usage}
\begin{verbatim}
drawBoxPlot(x, col = "black", fill = NULL, drawBox = T, drawDot = F,
  whiskerLty = 2, side = FALSE, lWidth = 1, capWidth = 0.25)
\end{verbatim}
\end{Usage}
%
\begin{Arguments}
\begin{ldescription}
\item[\code{x}] named list or data frame; \code{x\$at}, \code{x\$q2}, \code{x\$q4}, \code{x\$median}, \code{x\$min}, \code{x\$max} and \code{x\$width} must all be defined as numeric vectors in a named list or data.frame object.

\item[\code{col}] character; color vector that controls the line color.

\item[\code{fill}] character; color vector that determines the interior color of the box.

\item[\code{drawBox}] logical; draws the box and whiskers if set to \code{\LinkA{TRUE}{TRUE}}. The median line will be drawn regardless.

\item[\code{drawDot}] logical; draws a circle at the center of the median bar if set to \code{\LinkA{TRUE}{TRUE}}.

\item[\code{whiskerLty}] positive integer; sets the line type or \code{lty} option for plotting the wiskers.

\item[\code{side}] logical; if set to \code{\LinkA{TRUE}{TRUE}}, the box plots will be drawn horizontally.

\item[\code{lWidth}] positive integer; corresponds to lwd line width setting in base R.

\item[\code{capWidth}] numeric; size of the error bar cap relative to the box width.
\end{ldescription}
\end{Arguments}
%
\begin{Details}\relax
The input data frame \code{x} should include columns labels named 'at','q1',and 'q3', 'median', 'min', 'max' and 'width' in any order.
Each row will draw a box and whisker plot. The columns 'q1' and 'q3' refer to the 25\% and 75\% cumulative distribution values that bound the interquartile range.
If \code{side=TRUE} then the x and y axises are swapped to support horizontal plotting. The box and whiskers can be suppressed leaving only the median line and the optional center marker if so desired.
\end{Details}
%
\begin{SeeAlso}\relax
\code{\LinkA{boxplot}{boxplot}}, \code{\LinkA{niceBox}{niceBox}}
\end{SeeAlso}
%
\begin{Examples}
\begin{ExampleCode}
data(iris)
iData<-iris %>% group_by(Species) %>%
   summarize(median=median(Sepal.Length),min=min(Sepal.Length),max=max(Sepal.Length),
   q1=quantile(Sepal.Length)[2],q3=quantile(Sepal.Length)[4]) %>%
   bind_cols(at=c(1:3),width=c(.2,.3,.4))
plot(1,1,type="n",xlim=c(0,4),ylim=c(0,9))
drawBoxPlot(iData)
\end{ExampleCode}
\end{Examples}
\inputencoding{utf8}
\HeaderA{drawPoints}{draw dots for a dot plot}{drawPoints}
%
\begin{Description}\relax
takes a data frame of locations, values and an optional subgrouping factor and adds the data points to the active plot
\end{Description}
%
\begin{Usage}
\begin{verbatim}
drawPoints(x, type = "jitter", col = "black", size = 1, shape = 1,
  highlight = FALSE, width = 0.2, sidePlot = FALSE)
\end{verbatim}
\end{Usage}
%
\begin{Arguments}
\begin{ldescription}
\item[\code{x}] named list or data frame; \code{x\$at}, \code{x\$data} and \code{x\$pfact} (optional) should all be defined. These vectors are used to place the the point on the chart and determine the point level grouping (highlighting)

\item[\code{type}] character; determines how the points are arranged. Options are 'jitter', 'linear', 'beeswarm' and 'distribution'.

\item[\code{col}] character; vector of color names for plotting points. If length is greater than one it will be used for subgroups or will iterate over the groups.

\item[\code{size}] numeric; vector of cex values for point size. If length is greater than one it will be used for subgroups or will iterate over the groups.

\item[\code{shape}] numeric; vector determining point shapes (pch). If length is greater than one it will be used for subgroups or will iterate over the groups.

\item[\code{highlight}] logical; Should the point highlighting option be turned on (assumes that pfact is defined).

\item[\code{width}] numeric; determines how far points can deviate from the center category label for \code{type} options other than 'linear'.

\item[\code{sidePlot}] logical; plots dots for a horizontal rather than vertical axis.
\end{ldescription}
\end{Arguments}
%
\begin{Details}\relax
This function adds data points to a chart. These can be organized exactly as specified (linear), as a jitter cloud (jitter), as a waterfall plot (distribution) or as a swarm (beeswarm).
A factor labeled pfact can be included in \code{x} and used to highlight individual data points by setting \code{subGroup=\LinkA{TRUE}{TRUE}}. All graphic customization options can given as vectors and will be iterated over during plotting.
Note that the size/cex option can not be used to highlight pfact levels in a beeswarm plot and only the first element of the vector will be used.
\end{Details}
%
\begin{SeeAlso}\relax
\code{\LinkA{points}{points}}, \code{\LinkA{stripchart}{stripchart}}, \code{\LinkA{beeswarm}{beeswarm}}
\end{SeeAlso}
%
\begin{Examples}
\begin{ExampleCode}
data(iris)
boxplot(iris$Sepal.Length~iris$Species,ylab="Sepal Length")
iData<-data.frame(at=as.numeric(iris$Species),data=iris$Sepal.Length)
drawPoints(iData,type="jitter",col=c("red","blue","purple"))
\end{ExampleCode}
\end{Examples}
\inputencoding{utf8}
\HeaderA{errorBars}{draw custom error bars}{errorBars}
%
\begin{Description}\relax
Draws error bars with an optional cap at one end
\end{Description}
%
\begin{Usage}
\begin{verbatim}
errorBars(x, capType = c("none", "bar", "ball"), capSize = NULL,
  side = FALSE, col = "black", lType = 1, width = 1)
\end{verbatim}
\end{Usage}
%
\begin{Arguments}
\begin{ldescription}
\item[\code{x}] named list or data frame; \code{x\$start}, \code{x\$stop} and \code{x\$at} must all be defined as numeric vectors in a named list or data.frame object. In the case of a data frame, each row returns a single error bar.

\item[\code{capType}] character; can be set to 'none', 'bar', 'ball'. If set to 'bar' or ball, a round point or a line segment will be used to cap the end of the error bar.

\item[\code{capSize}] numeric; \code{capSize} is the distance that the cap extends away from the error bar. Set to \code{\LinkA{NULL}{NULL}} to suppress the cap regardless of the \code{capType} setting.

\item[\code{side}] logical; if set to true, the error bars will be drawn horizontally.

\item[\code{col}] color; a vector of line colors.

\item[\code{lType}] positive integer; corresponds to lty line type in base R.

\item[\code{width}] positive numeric; corresponds to lwd line width setting in base R.\bsl{}\#'
\end{ldescription}
\end{Arguments}
%
\begin{Details}\relax
The input data frame \code{x} should have columns labels 'at','start',and 'stop' with at determining the x-axis location and start and stop indicating the position of the segment on the y-axis. If \code{side=TRUE} then the x and y axises are swapped to support horizontal plotting. Each row of the data frame will produce one bar and an optional cap can be drawn at the 'stop' location.
\end{Details}
%
\begin{Examples}
\begin{ExampleCode}
data(iris)
iData<-iris %>% group_by(Species) %>%
   summarize(Average=mean(Sepal.Length),SD=sd(Sepal.Length))
barplot(iData$Average,ylim=c(0,10),names=levels(iris$Species),ylab="sepal length")
loc<-c(.7,1.9,3.1)
top<-iData$SD*2+iData$Average
bottom<-iData$SD*-2+iData$Average
errorBars(data.frame(at=loc,start=iData$Average,stop=top),capType="ball",capSize=2)
errorBars(data.frame(at=loc,start=iData$Average,stop=bottom),capType="ball",capSize=2)
\end{ExampleCode}
\end{Examples}
\inputencoding{utf8}
\HeaderA{facetSpacing}{Generate plotting locations for subgrouping data}{facetSpacing}
%
\begin{Description}\relax
\code{facetSpacing} generates a vector for the \code{at=} specification in functions for data sub-grouping
\end{Description}
%
\begin{Usage}
\begin{verbatim}
facetSpacing(subGroup, labels)
\end{verbatim}
\end{Usage}
%
\begin{Arguments}
\begin{ldescription}
\item[\code{subGroup}] positive integer; number of levels in the subgrouping factor

\item[\code{labels}] positive integer; number of levels in the primary factor
\end{ldescription}
\end{Arguments}
%
\begin{Details}\relax
\code{facetSpacing} takes the number factor levels from the primary and secodary grouping factors to generate a vector of positions for plotting subgrouped data for the nicePlots package.
The spacing assumes that each primary factor levels is plot on positive integers 1, 2, 3 etc.
For a primary factor at position \code{i} with \code{f} subgroup levels, the subgrouping comes from generating equally spaced intervals starting at \eqn{i-\frac{1}{2}+\frac{1}{f+1}}{} and ending at \eqn{i+\frac{1}{2}-\frac{1}{f+1}}{}. Simply put: \deqn{Spacing = \frac{1}{NSubGroups-1}}{}
\end{Details}
%
\begin{Value}
a numeric vector of where to plot the subgrouped data. Can be supplied to that \code{at=} option in plotting functions
\end{Value}
%
\begin{SeeAlso}\relax
\code{\LinkA{prepCategoryWindow}{prepCategoryWindow}}
\end{SeeAlso}
%
\begin{Examples}
\begin{ExampleCode}
boxplot(CNA$BM~ CNA$Status,border="white")
stripchart(CNA$BM~factor(paste0(CNA$Status,CNA$Sex)),add=T,at=facetSpacing(2,2))
\end{ExampleCode}
\end{Examples}
\inputencoding{utf8}
\HeaderA{formatPlotColors}{format a NicePlots color list}{formatPlotColors}
%
\begin{Description}\relax
To simplify code and user options, any color option not set by the user is added to the list and set to the default value.
\end{Description}
%
\begin{Usage}
\begin{verbatim}
formatPlotColors(plotColors, theme = NA)
\end{verbatim}
\end{Usage}
%
\begin{Arguments}
\begin{ldescription}
\item[\code{plotColors}] list; a named list of vectors of colors that set the color options for all NicePlot functions. Names left unspecified will be added and set to default values automatically.

\item[\code{theme}] list; A \code{NicePlots} plotColor list from a theme.
\end{ldescription}
\end{Arguments}
%
\begin{Details}\relax
The \code{NicePlots} plotColors object is a list of named color values/vectors. The options NicePlots colors include \code{bg} (background color), \code{marginBg} (color of area surrounding the plot), \code{guides} (guide lines for major tick-marks), \code{minorGuides} (guide lines for minor tick-marks)
\code{lines} (lines for box/bar plots etc.), \code{points} (plotting data points), \code{fill} (fill for box/bar plots etc.), \code{axis} (axis colors), \code{majorTick} (major tick-mark color),
\code{minorTick} (minor tick-mark color), \code{labels} (label colors), \code{subGroupLabels} (subgroup label colors), \code{rectCol} (inner quartile range box used in \code{\LinkA{niceVio}{niceVio}}), and \code{medianMarkerCol} (the median value marker used in \code{\LinkA{niceVio}{niceVio}}).
Any option not set be the user will be added to the list and set to the default in order to insure compatibility with downstream NicePlot functions.
If a theme is given, any option not set by the user will be set by the theme.
\end{Details}
%
\begin{Value}
a formated NicePlots color list.
\end{Value}
%
\begin{Examples}
\begin{ExampleCode}
myCols<-list(bg="lightgrey",fill=c("red","green","blue"),lines="darkgrey")
myCols<-formatPlotColors(myCols)
print(myCols)
\end{ExampleCode}
\end{Examples}
\inputencoding{utf8}
\HeaderA{makeColorMatrix}{Create a matrix of increasingly transparent colors}{makeColorMatrix}
%
\begin{Description}\relax
\code{makeColorMatrix} is a convenience function for plotting with transparent colors.
\end{Description}
%
\begin{Usage}
\begin{verbatim}
makeColorMatrix()
\end{verbatim}
\end{Usage}
%
\begin{Details}\relax
This function take no arguments, but generates rows corresponding to red, blue, green, gray, purple and gold with increasing transparency moving from left to right across the columns.
\end{Details}
%
\begin{Value}
A \code{6 x 5} matrix of colors.
\end{Value}
%
\begin{SeeAlso}\relax
\code{\LinkA{rainbow}{rainbow}}, \code{\LinkA{col2rgb}{col2rgb}}, \code{\LinkA{rgb}{rgb}}.
\end{SeeAlso}
%
\begin{Examples}
\begin{ExampleCode}
plot(1,1,col="white",xlim=c(0,10),ylim=c(0,10))
for(n in 1:6){rect(0:4,rep(8-n,5),1:5,rep(9-n,5),col=as.matrix(makeColorMatrix())[n,])}

#An example how it can be used in practice:
myData<-rnorm(600)
fact<-factor(c(rep("a",100),rep("b",100),rep("c",100),rep("d",100),rep("e",100),rep("f",100)))
plot(myData,col=makeColorMatrix()[fact,3])
\end{ExampleCode}
\end{Examples}
\inputencoding{utf8}
\HeaderA{makeLogTicks}{format a log scale axis}{makeLogTicks}
%
\begin{Description}\relax
Generates the location and labels for the major tick marks for a given log base transformation along with optional minor tick mark location.
\end{Description}
%
\begin{Usage}
\begin{verbatim}
makeLogTicks(dataRange, minorCount = 10, logScale = 2, axisText = c(NULL,
  NULL), expLabels = TRUE)
\end{verbatim}
\end{Usage}
%
\begin{Arguments}
\begin{ldescription}
\item[\code{dataRange}] numeric; a numeric vector with the min and max values for the data set prior to log transformation.

\item[\code{minorCount}] positive integer; the number of minor tick marks to be drawn between each major tick.

\item[\code{logScale}] numeric; the logarithm base to use for the log scale transformation.

\item[\code{axisText}] character; a length two character vector containing text to be prepend or append to the major tick labels, respectively.

\item[\code{expLabels}] logical; if set to \code{\LinkA{TRUE}{TRUE}}, the major labels will written as \eqn{logbase^{x}}{}. Otherwise the labels will correspond to the non-transformed values at that point.
\end{ldescription}
\end{Arguments}
%
\begin{Details}\relax
Base R does not have great visual queues to indicate when data is being plotted in log scale. This is a simple function takes the min and max of the untransformed data and  uses \code{\LinkA{axisTicks}{axisTicks}} from base R to determine the location of the major tick marks in the new scale. To better indicate that the graph is on a log scale, the major tick-marks are labeled in the untransformed values or expressed in as \eqn{logScale^{x}}{} when \code{expLabels=\LinkA{TRUE}{TRUE}}. The minor tick marks are drawn equidistant from each other between the major tick marks in the untransformed scale giving them shrinking appearance when rendered in log scale coordinates. This can help helps with the interpretation of data within the log scale and adds another visual indication that the data has been transformed. The value of \code{minorCount} gives number of minor ticks to be drawn between each pair of major tick-marks. \code{axisText} allows for symbols or units such as '
It is worth stressing again that the input values to dataRange are assumed to be raw values prior to log transformation. If log transformed values are given, the axis will be drawn correctly.
\end{Details}
%
\begin{Value}
a list with the following elements: major tick marks locations [[1]], major tick labels [[2]], minor tick mark locations [[3]].
\end{Value}
%
\begin{SeeAlso}\relax
\code{\LinkA{axisTicks}{axisTicks}}, \code{\LinkA{axis}{axis}}, \code{\LinkA{prepCategoryWindow}{prepCategoryWindow}}
\end{SeeAlso}
%
\begin{Examples}
\begin{ExampleCode}
plot(1:10,log(1:10,2),yaxt="n",ylab="")
majorTicks<-makeLogTicks(c(0,10),minorCount= 4,logScale=2, axisText=c("","mg"), expLabels=TRUE)
axis(side=2,lab=majorTicks[[2]],at=majorTicks[[1]],las=2)
axis(side = 2, at = majorTicks[[3]], labels = FALSE, tcl = -0.2)
\end{ExampleCode}
\end{Examples}
\inputencoding{utf8}
\HeaderA{makeNiceLegend}{Draw a nice plot legened}{makeNiceLegend}
%
\begin{Description}\relax
Draws a customizable legend in the margins based on factor levels.
\end{Description}
%
\begin{Usage}
\begin{verbatim}
makeNiceLegend(labels, title = "Legend", fontCol = "black", border = NULL,
  lineCol = NA, bg = NA, col = makeColorMatrix()[, 3], shape = "rect",
  size = 0.66, spacing = 0.2)
\end{verbatim}
\end{Usage}
%
\begin{Arguments}
\begin{ldescription}
\item[\code{labels}] character vector; The names of the levels decribed in the legend. Typically factor levels.

\item[\code{title}] character; The title of the legend. This defaults to "Legend" if unspecificed.

\item[\code{fontCol}] R color; Color of the legend text.

\item[\code{border}] R color; The color of the rectanglar border surrounding the legend. Defaults to \code{\LinkA{NULL}{NULL}} which supresses this feature

\item[\code{lineCol}] R color; The color of the line colors for the color key. Optional. Defaults to \code{\LinkA{NA}{NA}}.

\item[\code{bg}] R color; Sets the background color for the legend aread. Note that this can be distinct the the margin background.

\item[\code{col}] R color vector; A vector of colors determining the color of the color code boxes.

\item[\code{shape}] character; Determins if the color code is rectangles or circles. Valid ptions are "rect", "rectangle", "circ", or "circle". Not there is no funcitonal difference between the synonyms.

\item[\code{size}] numeric; Sets the legend font cex sizing.

\item[\code{spacing}] numeric; Determins the total amount of padding (sum of upper and lower padding) surrounding each line. in the legend in units of font line hight.
\end{ldescription}
\end{Arguments}
%
\begin{Details}\relax
This functions works with plot enviroment initializing functions such as \code{\LinkA{prepCategoryWindow}{prepCategoryWindow}}
to expand the right margin to accomodate a figure legend.
\end{Details}
%
\begin{SeeAlso}\relax
\code{\LinkA{legend}{legend}}, \code{\LinkA{prepCategoryWindow}{prepCategoryWindow}}, \code{\LinkA{niceBox}{niceBox}}, \code{\LinkA{niceDots}{niceDots}}, \code{\LinkA{niceBar}{niceBar}}, \code{\LinkA{niceVio}{niceVio}}
\end{SeeAlso}
%
\begin{Examples}
\begin{ExampleCode}
ToDo<-1

\end{ExampleCode}
\end{Examples}
\inputencoding{utf8}
\HeaderA{niceBar}{draw a bar plot}{niceBar}
%
\begin{Description}\relax
Aggregates data from a numeric vector or dataframe using up to three factors to draw a barplot with optional error bars.
\end{Description}
%
\begin{Usage}
\begin{verbatim}
niceBar(x, by = NULL, groupNames = NULL, aggFun = c("mean", "median",
  "none"), errFun = c("sd", "se", "range"), theme = basicTheme,
  stack = FALSE, main = NULL, sub = NULL, ylab = NULL,
  minorTick = FALSE, guides = TRUE, outliers = FALSE, width = 1,
  errorMultiple = 2, plotColors = list(bg = "open", fill = setAlpha("grey",
  0.8)), logScale = FALSE, trim = FALSE, axisText = c(NULL, NULL),
  showCalc = FALSE, calcType = "none", yLim = NULL,
  rotateLabels = FALSE, rotateY = TRUE, add = FALSE, minorGuides = NULL,
  extendTicks = TRUE, subGroup = FALSE, subGroupLabels = NULL,
  expLabels = FALSE, sidePlot = FALSE, errorBars = TRUE,
  errorCap = "ball", errorLineType = 1, capSize = 1.2, lWidth = 1.5,
  na.rm = FALSE, flipFacts = FALSE, verbose = FALSE, ...)
\end{verbatim}
\end{Usage}
%
\begin{Arguments}
\begin{ldescription}
\item[\code{x}] numeric vector or data frame; The input to \code{prepCategoryWindow} can be a numeric vector a  data frame of numeric vectors.

\item[\code{by}] factor or data frame of factors; used as the primary grouping factor and the factor levels will be used as group names if \code{groupNames} is not specified. If \code{by} is a data frame and \code{subGroup=\LinkA{TRUE}{TRUE}}, the second column is assumed to be a secondary grouping factor, breaking out the data into sub-categories within each major group determined by the levels of the first column.

\item[\code{groupNames}] character vector; overrides the factor levels of \code{by} to label the groups

\item[\code{aggFun}] character; Determines how the data is summarized by factor level. Valid options are \code{mean}, \code{median} or \code{none}.

\item[\code{errFun}] character; How the data spread is charactarized by the error bars. Valid options are \code{sd} (standard deviation), \code{se} (standard error of the mean) or \code{range}.

\item[\code{theme}] list object; Themes are are an optional way of storing graphical preset options that are compatible with all nicePlot graphing functions.

\item[\code{stack}] logical; Should one of the factors in \code{by} be used make a stacked bar plot. Note that this sort of analysis is nonsensical for many data sets.

\item[\code{main}] character; title for the graph which is supplied to the \code{main} argument.

\item[\code{sub}] character; subtitle for the graph which is supplied to the \code{sub} argument. If \code{\LinkA{NULL}{NULL}} and \code{showCalc=\LinkA{TRUE}{TRUE}} it will be used to display the output form \code{\LinkA{calcStats}{calcStats}}.

\item[\code{ylab}] character; y-axis label.

\item[\code{minorTick}] positive integer; number of minor tick-marks to draw between each pair of major ticks-marks.

\item[\code{guides}] logical; will draw guidelines at the major tick-marks if set to \code{\LinkA{TRUE}{TRUE}}. Color of the guidelines is determined by \code{plotColors\$guides}.

\item[\code{outliers}] positive numeric; number of interquartile ranges (IQR) past the Q1 (25\%) and Q3 (75\%) cumulative distribution values. Outliers are often defined as \eqn{1.5 \times IQR}{} and extreme outliers are more than \eqn{3 \times IQR}{} away from the inner 50\% data range.

\item[\code{width}] numeric; cex-like scaling factor controlling the width of the bars.

\item[\code{errorMultiple}] numeric; How many standard errors/deviations should be represented by the error bars.

\item[\code{plotColors}] list; a named list of vectors of colors that set the color options for all NicePlot functions. Names left unspecified will be added and set to default values automatically.

\item[\code{logScale}] positive numeric; the base for the for log scale data transformation calculated as \code{log(x+1,logScale)}.

\item[\code{trim}] positive numeric; passed to \code{threshold} argument of \code{\LinkA{quantileTrim}{quantileTrim}} if any data points are so extreme that they should be removed before plotting and downstream analysis. Set to \code{\LinkA{FALSE}{FALSE}} to disable.

\item[\code{axisText}] character; a length two character vector containing text to be prepended or appended to the major tick labels, respectively.

\item[\code{showCalc}] logical; if a p-value can be easily calculated for your data, it will be displayed using the \code{sub} annotation setting.

\item[\code{calcType}] character; should match one of 'none', 'wilcox', 'Tukey','t.test','anova' which will determine which, if any statistical test should be performed on the data.

\item[\code{yLim}] numeric vector; manually set the limits of the plotting area (eg. \code{yLim=c(min,max)}). Used to format the y-axis by default but will modify the x-axis if \code{side=\LinkA{TRUE}{TRUE}}.

\item[\code{rotateLabels}] logical; sets \code{las=2} for the x-axis category labels. Will affect y-axis if \code{side=\LinkA{TRUE}{TRUE}}. Note that this may not work well if long names or with subgrouped data.

\item[\code{rotateY}] logical; sets \code{las=2} for the y-axis major tick-mark labels. Will affect x-axis if \code{side=\LinkA{TRUE}{TRUE}}.

\item[\code{add}] logical; causes plotting to be added to the existing plot rather the start a new one.

\item[\code{minorGuides}] logical; draws guidelines at minor tick-marks

\item[\code{extendTicks}] logical; extends minor tick-marks past the first and last major tick to the edge of the graph provided there is enough room. Works for both log-scale and regular settings.

\item[\code{subGroup}] logical; use additional column in \code{by} to group the data within each level of the major factor.

\item[\code{subGroupLabels}] character vector; sets the labels used for the \code{subGroup} factor. Defaults to the levels of the factor.

\item[\code{expLabels}] logical; prints the major tick labels is \eqn{logScale^{x}}{} instead of the raw value

\item[\code{sidePlot}] logical; switches the axis to plot horizontally instead of vertically.

\item[\code{errorBars}] Logical; Should error bars be drawn. Defaults to true but is ignored if \code{stack=\LinkA{TRUE}{TRUE}}.

\item[\code{errorCap}] character; Determines the style for the ends of the error bars. Valid options are \code{ball}, \code{bar} or \code{none}.

\item[\code{errorLineType}] numeric; Sets \code{lty} line type for drawing the error bars.

\item[\code{capSize}] numeric; Controls the cex like scaling of the ball or width of the cap if they are drawn at the end of the error bars for the bar plot.

\item[\code{lWidth}] numeric; Line width (lwd) for drawing the bar plot.

\item[\code{na.rm}] logical; Should \code{NA}s be removed from the data set? Both data input and the factor input from \code{by} with be checked.

\item[\code{flipFacts}] logical; When a dataframe of values is given, column names are used as a secondary grouping factor by default. Setting \code{flipFacts=\LinkA{TRUE}{TRUE}} makes the column names the primary factor and \code{by} the secondary factor.

\item[\code{verbose}] logical; Prints summary and p-value calculations to the screen. All data is silently by the function returned either way.

\item[\code{...}] additional options for S3 method variants.
\end{ldescription}
\end{Arguments}
%
\begin{Details}\relax
This bar plot function allows for standard barplot features but with error bars, the ability
summaryize dataframes into bar plots with median/mean values, sort by bar hight for waterfall plots,
color bars based on interquartile outlier detection and more. Barplots can be clustered by a secondary factor
or if a dataframe is passed to \code{x} the input values of multiple measurments (dataframe columns) can be
clustered together by the primary factor. As with \code{\LinkA{niceBox}{niceBox}}, \code{\LinkA{niceDots}{niceDots}}
and \code{\LinkA{niceVio}{niceVio}}, \code{by} can be a factor or a dataframe factors for forming subgroups.

For most data this would be nonsensical but if you data is say store profits by goods by region one could group by region (first)
\end{Details}
%
\begin{SeeAlso}\relax
\code{\LinkA{vioplot}{vioplot}}, \code{\LinkA{boxplot}{boxplot}}, \code{\LinkA{niceBox}{niceBox}}, \code{\LinkA{beeswarm}{beeswarm}}, \code{\LinkA{prepCategoryWindow}{prepCategoryWindow}}
\end{SeeAlso}
%
\begin{Examples}
\begin{ExampleCode}
data(mtcars)
Groups<-data.frame(Cyl=factor(mtcars$cyl),Gear=factor(mtcars$gear))
niceBar(mtcars$mpg,by=Groups,subGroup=TRUE,yLim=c(0,45),main="MpG by Cylinders and Gear")

\end{ExampleCode}
\end{Examples}
\inputencoding{utf8}
\HeaderA{niceBox}{draw a box plot}{niceBox}
%
\begin{Description}\relax
draws a box plot with optional scatter plot overlays, subgrouping options and log scale support.
\end{Description}
%
\begin{Usage}
\begin{verbatim}
niceBox(x, by = NULL, groupNames = NULL, main = NULL, sub = NULL,
  ylab = NULL, theme = basicTheme, minorTick = FALSE, guides = TRUE,
  outliers = 1.5, pointSize = 1, width = 1, pointShape = 16,
  plotColors = list(bg = "open"), logScale = FALSE, trim = FALSE,
  pointMethod = "jitter", axisText = c(NULL, NULL), showCalc = FALSE,
  calcType = "none", drawBox = TRUE, yLim = NULL, rotateLabels = FALSE,
  rotateY = FALSE, add = FALSE, minorGuides = NULL, extendTicks = TRUE,
  subGroup = FALSE, subGroupLabels = NULL, expLabels = TRUE,
  sidePlot = FALSE, drawPoints = TRUE, pointHighlights = FALSE,
  drawCenterDot = !drawPoints, pointLaneWidth = 0.7, flipFacts = FALSE,
  na.rm = FALSE, verbose = FALSE, legend = FALSE, ...)
\end{verbatim}
\end{Usage}
%
\begin{Arguments}
\begin{ldescription}
\item[\code{x}] numeric vector or data frame; The input to \code{prepCategoryWindow} can be a numeric vector a  data frame of numeric vectors.

\item[\code{by}] factor or data frame of factors; used as the primary grouping factor and the factor levels will be used as group names if \code{groupNames} is not specified. If \code{by} is a data frame and \code{subGroup=\LinkA{TRUE}{TRUE}}, the second column is assumed to be a secondary grouping factor, breaking out the data into sub-categories within each major group determined by the levels of the first column.

\item[\code{groupNames}] character vector; overrides the factor levels of \code{by} to label the groups

\item[\code{main}] character; title for the graph which is supplied to the \code{main} argument.

\item[\code{sub}] character; subtitle for the graph which is supplied to the \code{sub} argument. If \code{\LinkA{NULL}{NULL}} and \code{showCalc=\LinkA{TRUE}{TRUE}} it will be used to display the output form \code{\LinkA{calcStats}{calcStats}}.

\item[\code{ylab}] character; y-axis label.

\item[\code{theme}] list object; Themes are are an optional way of storing graphical preset options that are compatible with all nicePlot graphing functions.

\item[\code{minorTick}] positive integer; number of minor tick-marks to draw between each pair of major ticks-marks.

\item[\code{guides}] logical; will draw guidelines at the major tick-marks if set to \code{\LinkA{TRUE}{TRUE}}. Color of the guidelines is determined by \code{plotColors\$guides}.

\item[\code{outliers}] positive numeric; number of interquartile ranges (IQR) past the Q1 (25\%) and Q3 (75\%) cumulative distribution values. Outliers are often defined as \eqn{1.5 \times IQR}{} and extreme outliers are more than \eqn{3 \times IQR}{} away from the inner 50\% data range.

\item[\code{pointSize}] positive integer; sets the cex multiplier for point size.

\item[\code{width}] numeric; scaling factor controlling the width of the boxes.

\item[\code{pointShape}] positive integer; sets pty for plotting data points. Can be a vector to support additional graphical customization.

\item[\code{plotColors}] list; a named list of vectors of colors that set the color options for all NicePlot functions. Names left unspecified will be added and set to default values automatically.

\item[\code{logScale}] positive numeric; the base for the for log scale data transformation calculated as \code{log(x+1,logScale)}.

\item[\code{trim}] positive numeric; passed to \code{threshold} argument of \code{\LinkA{quantileTrim}{quantileTrim}} if any data points are so extreme that they should be removed before plotting and downstream analysis. Set to \code{\LinkA{FALSE}{FALSE}} to disable.

\item[\code{pointMethod}] character; method to be used for ploting dots. Can be set to "jitter", "linear", "beeswarm" or "distribution".

\item[\code{axisText}] character; a length two character vector containing text to be prepended or appended to the major tick labels, respectively.

\item[\code{showCalc}] logical; if a p-value can be easily calculated for your data, it will be displayed using the \code{sub} annotation setting.

\item[\code{calcType}] character; should match one of 'none', 'wilcox', 'Tukey','t.test','anova' which will determine which, if any statistical test should be performed on the data.

\item[\code{drawBox}] logical; should the boxes be drawn. The median bar will be drawn regardless.

\item[\code{yLim}] numeric vector; manually set the limits of the plotting area (eg. \code{yLim=c(min,max)}). Used to format the y-axis by default but will modify the x-axis if \code{side=\LinkA{TRUE}{TRUE}}.

\item[\code{rotateLabels}] logical; sets \code{las=2} for the x-axis category labels. Will affect y-axis if \code{side=\LinkA{TRUE}{TRUE}}. Note that this may not work well if long names or with subgrouped data.

\item[\code{rotateY}] logical; sets \code{las=2} for the y-axis major tick-mark labels. Will affect x-axis if \code{side=\LinkA{TRUE}{TRUE}}.

\item[\code{add}] logical; causes plotting to be added to the existing plot rather the start a new one.

\item[\code{minorGuides}] logical; draws guidelines at minor tick-marks

\item[\code{extendTicks}] logical; extends minor tick-marks past the first and last major tick to the edge of the graph provided there is enough room. Works for both log-scale and regular settings.

\item[\code{subGroup}] logical; use additional column in \code{by} to group the data within each level of the major factor.

\item[\code{subGroupLabels}] character vector; sets the labels used for the \code{subGroup} factor. Defaults to the levels of the factor.

\item[\code{expLabels}] logical; prints the major tick labels is \eqn{logScale^{x}}{} instead of the raw value

\item[\code{sidePlot}] logical; switches the axis to plot horizontally instead of vertically.

\item[\code{drawPoints}] logical; draws a dot plot overlay of the data for each box.

\item[\code{pointHighlights}] logical; will use additional factors in \code{by} to highlight points in the dot plot

\item[\code{drawCenterDot}] logical; draws a circle at the middle of the median bar. Will be turned off by default of drawPoints is set to \code{\LinkA{TRUE}{TRUE}}.

\item[\code{pointLaneWidth}] numeric; This controls how far data point dots can move along the categorical axis when plotting. Used for \code{pointMethod} options 'jitter', 'beeswarm', and 'distribution'.

\item[\code{flipFacts}] logical; When a dataframe of values is given, column names are used as a secondary grouping factor by default. Setting \code{flipFacts=\LinkA{TRUE}{TRUE}} makes the column names the primary factor and \code{by} the secondary factor.

\item[\code{na.rm}] logical; Should \code{NA}s be removed from the data set? Both data input and the factor input from \code{by} with be checked.

\item[\code{verbose}] logical; Prints summary and p-value calculations to the screen. All data is silently by the function returned either way.

\item[\code{legend}] logical/character; Draw a legend in the plot margins. If a character string is given it will overide the factor name default for the legend title.

\item[\code{...}] additional options for S3 method variants
\end{ldescription}
\end{Arguments}
%
\begin{Details}\relax
This box plot function offers extensive log scale support, outlier detection, data point overlay options, data subsetting with a secondary factor, and data point highlighting with a tertiary factor.
The complicated part of using this function is handling its many options. A wrapper function to set up and run it with preset options may be a good idea if you are using it along. The function \code{\LinkA{niceDots}{niceDots}} is an example of this.
Briefly put, the \code{by} argument can be a data frame of factors and the function will  work through the columns in order as needed.
If \code{x} is a numeric vector, then \code{by} should be a factor to group it into categories. If \code{by} is a data frame of factors and \code{subGroup=\LinkA{TRUE}{TRUE}}, then the first column for \code{by}
is used as the grouping factor and the second column is used as the sub-grouping factor. If \code{pointHighlights==\LinkA{TRUE}{TRUE}}, and \code{subGroup=\LinkA{TRUE}{TRUE}}, the the third column of \code{by}
is used to highlight points data point overlay (assuming \code{drawPoints=\LinkA{TRUE}{TRUE}}). If \code{subGroup=\LinkA{FALSE}{FALSE}} and \code{subGroup=\LinkA{TRUE}{TRUE}}, then the second column of \code{by} is used to control
the point highlighting. If \code{x} itself is a data frame of numeric vectors, \code{subGroup} is automatically set to false and each column of \code{x} is plotted like a sub-group and grouped
by the first column of \code{by}. Data point highlighting with \code{pointHighlights=\LinkA{TRUE}{TRUE}} can still be used when \code{x} is a data frame and the highlighting factor will be drawn from the second column of \code{by}.
Please note that the p-values can not always be calculated and are for general exploratory use only. More careful analysis is necessary to determine statistical significance.
This function is as S3 generic and can be extended to provide class specific functionality. To further facilitate data exploration, outputs from statistical testing and data set summaries
are printed to the console.
\end{Details}
%
\begin{SeeAlso}\relax
\code{\LinkA{boxplot}{boxplot}}, \code{\LinkA{beeswarm}{beeswarm}}, \code{\LinkA{quantileTrim}{quantileTrim}}, \code{\LinkA{prepCategoryWindow}{prepCategoryWindow}}
\end{SeeAlso}
%
\begin{Examples}
\begin{ExampleCode}
data(iris)
mCols<-makeColorMatrix()
myCols<-list(fill=c(mCols[1,3],mCols[2,3],mCols[3,3]),lines="darkblue")
Lab<-"Sepal Length"
niceBox(iris$Sepal.Length,iris$Species,minorTick=4,showCalc=TRUE,
    calcType="anova",ylab=Lab,main="Sepal Length by Species",plotColors=myCols)


plot(density(iris$Petal.Length))
lengthFact<-factor(iris$Petal.Length>2.82,labels=c("short","long"))


Title<-"Sepal Length by Species and Petal Length"
factorFrame<-data.frame(Species=iris$Species,PetalLength=lengthFact)
niceBox(iris$Sepal.Length, by=factorFrame, minorTick=4,subGroup=TRUE,
    ylab=Lab,main=Title,plotColors=myCols)
\end{ExampleCode}
\end{Examples}
\inputencoding{utf8}
\HeaderA{niceDots}{draw a dot plot}{niceDots}
%
\begin{Description}\relax
draws a categorical dot plot with optional data highlighting and log scale support.
\end{Description}
%
\begin{Usage}
\begin{verbatim}
niceDots(x, by = NULL, groupNames = NULL, main = NULL, sub = NULL,
  ylab = NULL, minorTick = FALSE, theme = basicTheme, guides = TRUE,
  outliers = 1.5, pointSize = 1, width = 1, pointShape = 1,
  plotColors = list(bg = "open"), logScale = FALSE, trim = FALSE,
  pointMethod = NULL, axisText = c(NULL, NULL), showCalc = FALSE,
  calcType = "none", yLim = NULL, rotateLabels = FALSE, rotateY = FALSE,
  add = FALSE, minorGuides = NULL, extendTicks = TRUE, subGroup = FALSE,
  subGroupLabels = NULL, expLabels = TRUE, sidePlot = FALSE,
  pointHighlights = FALSE, pointLaneWidth = 1, na.rm = FALSE,
  flipFacts = FALSE, verbose = FALSE, legend = FALSE, ...)
\end{verbatim}
\end{Usage}
%
\begin{Arguments}
\begin{ldescription}
\item[\code{x}] numeric vector or data frame; The input to \code{prepCategoryWindow} can be a numeric vector a  data frame of numeric vectors.

\item[\code{by}] factor or data frame of factors; used as the primary grouping factor and the factor levels will be used as group names if \code{groupNames} is not specified. If \code{by} is a data frame and \code{subGroup=\LinkA{TRUE}{TRUE}}, the second column is assumed to be a secondary grouping factor, breaking out the data into sub-categories within each major group determined by the levels of the first column.

\item[\code{groupNames}] character vector; overrides the factor levels of \code{by} to label the groups

\item[\code{main}] character; title for the graph which is supplied to the \code{main} argument.

\item[\code{sub}] character; subtitle for the graph which is supplied to the \code{sub} argument. If \code{\LinkA{NULL}{NULL}} and \code{showCalc=\LinkA{TRUE}{TRUE}} it will be used to display the output form \code{\LinkA{calcStats}{calcStats}}.

\item[\code{ylab}] character; y-axis label.

\item[\code{minorTick}] positive integer; number of minor tick-marks to draw between each pair of major ticks-marks.

\item[\code{theme}] list object; Themes are are an optional way of storing graphical preset options that are compatible with all nicePlot graphing functions.

\item[\code{guides}] logical; will draw guidelines at the major tick-marks if set to \code{\LinkA{TRUE}{TRUE}}. Color of the guidelines is determined by \code{plotColors\$guides}.

\item[\code{outliers}] positive numeric; number of interquartile ranges (IQR) past the Q1 (25\%) and Q3 (75\%) cumulative distribution values. Outliers are often defined as \eqn{1.5 \times IQR}{} and extreme outliers are more than \eqn{3 \times IQR}{} away from the inner 50\% data range.

\item[\code{pointSize}] positive integer; sets the cex multiplier for point size.

\item[\code{width}] numeric; scaling factor controlling the width of the boxes.

\item[\code{pointShape}] positive integer; sets pty for plotting data points. Can be a vector to support additional graphical customization.

\item[\code{plotColors}] list; a named list of vectors of colors that set the color options for all NicePlot functions. Names left unspecified will be added and set to default values automatically.

\item[\code{logScale}] positive numeric; the base for the for log scale data transformation calculated as \code{log(x+1,logScale)}.

\item[\code{trim}] positive numeric; passed to \code{threshold} argument of \code{\LinkA{quantileTrim}{quantileTrim}} if any data points are so extreme that they should be removed before plotting and downstream analysis. Set to \code{\LinkA{FALSE}{FALSE}} to disable.

\item[\code{pointMethod}] character; method to be used for ploting dots. Can be set to "jitter", "linear", "beeswarm" or "distribution".

\item[\code{axisText}] character; a length two character vector containing text to be prepended or appended to the major tick labels, respectively.

\item[\code{showCalc}] logical; if a p-value can be easily calculated for your data, it will be displayed using the \code{sub} annotation setting.

\item[\code{calcType}] character; should match one of 'none', 'wilcox', 'Tukey','t.test','anova' which will determine which, if any statistical test should be performed on the data.

\item[\code{yLim}] numeric vector; manually set the limits of the plotting area (eg. \code{yLim=c(min,max)}). Used to format the y-axis by default but will modify the x-axis if \code{side=\LinkA{TRUE}{TRUE}}.

\item[\code{rotateLabels}] logical; sets \code{las=2} for the x-axis category labels. Will affect y-axis if \code{side=\LinkA{TRUE}{TRUE}}. Note that this may not work well if long names or with subgrouped data.

\item[\code{rotateY}] logical; sets \code{las=2} for the y-axis major tick-mark labels. Will affect x-axis if \code{side=\LinkA{TRUE}{TRUE}}.

\item[\code{add}] logical; causes plotting to be added to the existing plot rather the start a new one.

\item[\code{minorGuides}] logical; draws guidelines at minor tick-marks

\item[\code{extendTicks}] logical; extends minor tick-marks past the first and last major tick to the edge of the graph provided there is enough room. Works for both log-scale and regular settings.

\item[\code{subGroup}] logical; use additional column in \code{by} to group the data within each level of the major factor.

\item[\code{subGroupLabels}] character vector; sets the labels used for the \code{subGroup} factor. Defaults to the levels of the factor.

\item[\code{expLabels}] logical; prints the major tick labels is \eqn{logScale^{x}}{} instead of the raw value

\item[\code{sidePlot}] logical; switches the axis to plot horizontally instead of vertically.

\item[\code{pointHighlights}] logical; will use additional factors in \code{by} to highlight points in the dot plot

\item[\code{pointLaneWidth}] numeric; This controls how far data point dots can move along the categorical axis when plotting. Used for \code{pointMethod} options 'jitter', 'beeswarm', and 'distribution'.

\item[\code{na.rm}] logical; Should \code{NA}s be removed from the data set? Both data input and the factor input from \code{by} with be checked.

\item[\code{flipFacts}] logical; When a dataframe of values is given, column names are used as a secondary grouping factor by default. Setting \code{flipFacts=\LinkA{TRUE}{TRUE}} makes the column names the primary factor and \code{by} the secondary factor.

\item[\code{verbose}] logical; Prints summary and p-value calculations to the screen. All data is silently by the function returned either way.

\item[\code{legend}] logical/character; Draw a legend in the plot margins. If a character string is given it will overide the factor name default for the legend title.

\item[\code{...}] additional options for S3 method variants
\end{ldescription}
\end{Arguments}
%
\begin{Details}\relax
This is a wrapper function for \code{\LinkA{niceBox}{niceBox}} that just plots the points with no box distribution data. data point overlay options, data subsetting with a secondary factor, and data point highlighting with a tertiary factor.
The complicated part of using this function is handling its many options. A wrapper function to set up and run it with preset options may be a good idea if you are using it along. The function \code{\LinkA{niceDots}{niceDots}} is an example of this.
Briefly put, the \code{by} argument can be a data frame of factors and the function will  work through the columns in order as needed.
If \code{x} is a numeric vector, then \code{by} should be a factor to group it into categories. If \code{by} is a data frame of factors and \code{subGroup=\LinkA{TRUE}{TRUE}}, then the first column for \code{by}
is used as the grouping factor and the second column is used as the sub-grouping factor. If \code{pointHighlights==\LinkA{TRUE}{TRUE}}, and \code{subGroup=\LinkA{TRUE}{TRUE}}, the the third column of \code{by}
is used to highlight points data point overlay (assuming \code{drawPoints=\LinkA{TRUE}{TRUE}}). If \code{subGroup=\LinkA{FALSE}{FALSE}} and \code{subGroup=\LinkA{TRUE}{TRUE}}, then the second column of \code{by} is used to control
the point highlighting. If \code{x} itself is a data frame of numeric vectors, \code{subGroup} is automatically set to false and each column of \code{x} is plotted like a sub-group and grouped
by the first column of \code{by}. Data point highlighting with \code{pointHighlights=\LinkA{TRUE}{TRUE}} can still be used when \code{x} is a data frame and the highlighting factor will be drawn from the second column of \code{by}.
Please note that the p-values can not always be calculated and are for general exploratory use only. More careful analysis is necessary to determine statistical significance.
This function is as S3 generic and can be extended to provide class specific functionality.
To further facilitate data exploration, outputs from statistical testing and data set summaries
are printed to the console.
\end{Details}
%
\begin{SeeAlso}\relax
\code{\LinkA{stripchart}{stripchart}}, \code{\LinkA{beeswarm}{beeswarm}}, \code{\LinkA{quantileTrim}{quantileTrim}}, \code{\LinkA{prepCategoryWindow}{prepCategoryWindow}}, \code{\LinkA{niceBox}{niceBox}}
\end{SeeAlso}
%
\begin{Examples}
\begin{ExampleCode}
data(iris)
mCols<-makeColorMatrix()
myCols<-list(fill=mCols[1:3,3],lines="darkblue")
niceDots(iris$Sepal.Length,iris$Species,minorTick=4,showCalc=TRUE,calcType="anova",
    ylab="Sepal Length",main="Sepal Length by Species",plotColors=myCols)

\end{ExampleCode}
\end{Examples}
\inputencoding{utf8}
\HeaderA{niceVio}{draw a violin plot}{niceVio}
%
\begin{Description}\relax
draws a violin plot with optional scatter plot overlays, subgrouping options and log scale support.
\end{Description}
%
\begin{Usage}
\begin{verbatim}
niceVio(x, by = NULL, h = NULL, groupNames = NULL, main = NULL,
  sub = NULL, ylab = NULL, minorTick = FALSE, guides = TRUE,
  theme = basicTheme, outliers = 1.5, pointSize = 1, width = 1,
  pointShape = 16, plotColors = list(bg = "open"), logScale = FALSE,
  trim = FALSE, pointMethod = "jitter", axisText = c(NULL, NULL),
  showCalc = FALSE, calcType = "none", drawBox = TRUE, yLim = NULL,
  rotateLabels = FALSE, rotateY = FALSE, add = FALSE,
  minorGuides = NULL, extendTicks = TRUE, subGroup = FALSE,
  subGroupLabels = NULL, expLabels = TRUE, sidePlot = FALSE,
  drawPoints = TRUE, pointHighlights = FALSE, pointLaneWidth = 0.7,
  flipFacts = FALSE, na.rm = FALSE, verbose = FALSE, legend = FALSE,
  ...)
\end{verbatim}
\end{Usage}
%
\begin{Arguments}
\begin{ldescription}
\item[\code{x}] numeric vector or data frame; The input to \code{prepCategoryWindow} can be a numeric vector a  data frame of numeric vectors.

\item[\code{by}] factor or data frame of factors; used as the primary grouping factor and the factor levels will be used as group names if \code{groupNames} is not specified. If \code{by} is a data frame and \code{subGroup=\LinkA{TRUE}{TRUE}}, the second column is assumed to be a secondary grouping factor, breaking out the data into sub-categories within each major group determined by the levels of the first column.

\item[\code{h}] numeric; Used to override the \code{h} hight of density estimator setting in \code{\LinkA{vioplot}{vioplot}}. Default value is \code{\LinkA{NULL}{NULL}}.

\item[\code{groupNames}] character vector; overrides the factor levels of \code{by} to label the groups

\item[\code{main}] character; title for the graph which is supplied to the \code{main} argument.

\item[\code{sub}] character; subtitle for the graph which is supplied to the \code{sub} argument. If \code{\LinkA{NULL}{NULL}} and \code{showCalc=\LinkA{TRUE}{TRUE}} it will be used to display the output form \code{\LinkA{calcStats}{calcStats}}.

\item[\code{ylab}] character; y-axis label.

\item[\code{minorTick}] positive integer; number of minor tick-marks to draw between each pair of major ticks-marks.

\item[\code{guides}] logical; will draw guidelines at the major tick-marks if set to \code{\LinkA{TRUE}{TRUE}}. Color of the guidelines is determined by \code{plotColors\$guides}.

\item[\code{theme}] list object; Themes are are an optional way of storing graphical preset options that are compatible with all nicePlot graphing functions.

\item[\code{outliers}] positive numeric; number of interquartile ranges (IQR) past the Q1 (25\%) and Q3 (75\%) cumulative distribution values. Outliers are often defined as \eqn{1.5 \times IQR}{} and extreme outliers are more than \eqn{3 \times IQR}{} away from the inner 50\% data range.

\item[\code{pointSize}] positive integer; sets the cex multiplier for point size.

\item[\code{width}] numeric; scaling factor controlling the width of the violins.

\item[\code{pointShape}] positive integer; sets pty for plotting data points. Can be a vector to support additional graphical customization.

\item[\code{plotColors}] list; a named list of vectors of colors that set the color options for all NicePlot functions. Names left unspecified will be added and set to default values automatically.

\item[\code{logScale}] positive numeric; the base for the for log scale data transformation calculated as \code{log(x+1,logScale)}.

\item[\code{trim}] positive numeric; passed to \code{threshold} argument of \code{\LinkA{quantileTrim}{quantileTrim}} if any data points are so extreme that they should be removed before plotting and downstream analysis. Set to \code{\LinkA{FALSE}{FALSE}} to disable.

\item[\code{pointMethod}] character; method to be used for ploting dots. Can be set to "jitter", "linear", "beeswarm" or "distribution".

\item[\code{axisText}] character; a length two character vector containing text to be prepended or appended to the major tick labels, respectively.

\item[\code{showCalc}] logical; if a p-value can be easily calculated for your data, it will be displayed using the \code{sub} annotation setting.

\item[\code{calcType}] character; should match one of 'none', 'wilcox', 'Tukey','t.test','anova' which will determine which, if any statistical test should be performed on the data.

\item[\code{drawBox}] logical; should the interquartile boxes be drawn.

\item[\code{yLim}] numeric vector; manually set the limits of the plotting area (eg. \code{yLim=c(min,max)}). Used to format the y-axis by default but will modify the x-axis if \code{side=\LinkA{TRUE}{TRUE}}.

\item[\code{rotateLabels}] logical; sets \code{las=2} for the x-axis category labels. Will affect y-axis if \code{side=\LinkA{TRUE}{TRUE}}. Note that this may not work well if long names or with subgrouped data.

\item[\code{rotateY}] logical; sets \code{las=2} for the y-axis major tick-mark labels. Will affect x-axis if \code{side=\LinkA{TRUE}{TRUE}}.

\item[\code{add}] logical; causes plotting to be added to the existing plot rather the start a new one.

\item[\code{minorGuides}] logical; draws guidelines at minor tick-marks

\item[\code{extendTicks}] logical; extends minor tick-marks past the first and last major tick to the edge of the graph provided there is enough room. Works for both log-scale and regular settings.

\item[\code{subGroup}] logical; use additional column in \code{by} to group the data within each level of the major factor.

\item[\code{subGroupLabels}] character vector; sets the labels used for the \code{subGroup} factor. Defaults to the levels of the factor.

\item[\code{expLabels}] logical; prints the major tick labels is \eqn{logScale^{x}}{} instead of the raw value

\item[\code{sidePlot}] logical; switches the axis to plot horizontally instead of vertically.

\item[\code{drawPoints}] logical; draws a dot plot overlay of the data for each box

\item[\code{pointHighlights}] logical; will use additional factors in \code{by} to highlight points in the dot plot

\item[\code{pointLaneWidth}] numeric; This controls how far data point dots can move along the categorical axis when plotting. Used for \code{pointMethod} options 'jitter', 'beeswarm', and 'distribution'.

\item[\code{flipFacts}] logical; When a dataframe of values is given, column names are used as a secondary grouping factor by default. Setting \code{flipFacts=\LinkA{TRUE}{TRUE}} makes the column names the primary factor and \code{by} the secondary factor.

\item[\code{na.rm}] logical; Should \code{NA}s be removed from the data set? Both data input and the factor input from \code{by} with be checked.

\item[\code{verbose}] logical; Prints summary and p-value calculations to the screen. All data is silently by the function returned either way.

\item[\code{legend}] logical/character; if not equal to \code{\LinkA{FALSE}{FALSE}} with cause a legend to be drawn in the margins. If set to a character string instead of a logical value, the string will be used as the legend title insteas of the factor column name from \code{by}.

\item[\code{...}] additional options for S3 method variants
\end{ldescription}
\end{Arguments}
%
\begin{Details}\relax
This violin plot function offers extensive log scale support, outlier detection, data point overlay options, data subsetting with a secondary factor, and data point highlighting with a tertiary factor.
The complicated part of using this function is handling its many options. A wrapper function to set up and run it with preset options may be a good idea if you are using it along. The function \code{\LinkA{niceDots}{niceDots}} is an example of this.
Briefly put, the \code{by} argument can be a data frame of factors and the function will  work through the columns in order as needed.
If \code{x} is a numeric vector, then \code{by} should be a factor to group it into categories. If \code{by} is a data frame of factors and \code{subGroup=\LinkA{TRUE}{TRUE}}, then the first column for \code{by}
is used as the grouping factor and the second column is used as the sub-grouping factor. If \code{pointHighlights==\LinkA{TRUE}{TRUE}}, and \code{subGroup=\LinkA{TRUE}{TRUE}}, the the third column of \code{by}
is used to highlight points data point overlay (assuming \code{drawPoints=\LinkA{TRUE}{TRUE}}). If \code{subGroup=\LinkA{FALSE}{FALSE}} and \code{subGroup=\LinkA{TRUE}{TRUE}}, then the second column of \code{by} is used to control
the point highlighting. If \code{x} itself is a data frame of numeric vectors, \code{subGroup} is automatically set to false and each column of \code{x} is plotted like a sub-group and grouped
by the first column of \code{by}. Data point highlighting with \code{pointHighlights=\LinkA{TRUE}{TRUE}} can still be used when \code{x} is a data frame and the highlighting factor will be drawn from the second column of \code{by}.
Please note that the p-values can not always be calculated and are for general exploratory use only. More careful analysis is necessary to determine statistical significance.
This function is as S3 generic and can be extended to provide class specific functionality. To further facilitate data exploration, outputs from statistical testing and data set summaries
are printed to the console.
\end{Details}
%
\begin{SeeAlso}\relax
\code{\LinkA{vioplot}{vioplot}}, \code{\LinkA{boxplot}{boxplot}}, \code{\LinkA{niceBox}{niceBox}}, \code{\LinkA{beeswarm}{beeswarm}}, \code{\LinkA{prepCategoryWindow}{prepCategoryWindow}}
\end{SeeAlso}
%
\begin{Examples}
\begin{ExampleCode}
data(iris)
mCols<-makeColorMatrix()
myCols<-list(fill=c(mCols[1,3],mCols[2,3],mCols[3,3]),lines="darkblue")
Lab<-"Sepal Length"
niceVio(iris$Sepal.Length,by=iris$Species,minorTick=4,showCalc=TRUE,
    calcType="anova",ylab=Lab,main="Sepal Length by Species",plotColors=myCols)


plot(density(iris$Petal.Length))
lengthFact<-factor(iris$Petal.Length>2.82,labels=c("short","long"))


Title<-"Sepal Length by Species and Petal Length"
factorFrame<-data.frame(Species=iris$Species,PetalLength=lengthFact)
niceVio(iris$Sepal.Length, by=factorFrame, minorTick=4,subGroup=TRUE,
    ylab=Lab,main=Title,plotColors=myCols)
\end{ExampleCode}
\end{Examples}
\inputencoding{utf8}
\HeaderA{prepCategoryWindow}{prepare a plotting environment for categorical data such as bar plots or box plots}{prepCategoryWindow}
%
\begin{Description}\relax
takes untransformed data and draws the x and y axis with support of subgrouping data within factors, log transformation and outlier trimming.
\end{Description}
%
\begin{Usage}
\begin{verbatim}
prepCategoryWindow(x, by = NULL, groupNames = levels(by),
  minorTick = FALSE, guides = TRUE, yLim = NULL, rotateLabels = FALSE,
  rotateY = TRUE, theme = NA, plotColors = if (is.na(theme)) {     list(bg
  = "open", guides = "black", lines = "gray22", points = "darkgrey", fill =
  "white") } else {     theme$plotColors }, trim = FALSE, logScale = FALSE,
  axisText = c(NULL, NULL), minorGuides = FALSE, extendTicks = F,
  subGroup = FALSE, expLabels = TRUE, sidePlot = FALSE,
  subGroupLabels = NULL, strictLimits = F, legend = FALSE,
  pointHighlights = FALSE)
\end{verbatim}
\end{Usage}
%
\begin{Arguments}
\begin{ldescription}
\item[\code{x}] numeric vector or data frame; The input to \code{prepCategoryWindow} can be a numeric vector a  data frame of numeric vectors.

\item[\code{by}] factor or data frame of factors; used as the primary grouping factor and the factor levels will be used as group names if \code{groupNames} is not specified. If \code{by} is a data frame and \code{subGroup=\LinkA{TRUE}{TRUE}}, the second column is assumed to be a secondary grouping factor, breaking out the data into sub-categories within each major group determined by the levels of the first column.

\item[\code{groupNames}] character vector; overrides the factor levels of \code{by} to label the groups

\item[\code{minorTick}] positive integer; number of minor tick-marks to draw between each pair of major ticks-marks.

\item[\code{guides}] logical; will draw guidelines at the major tick-marks if set to \code{\LinkA{TRUE}{TRUE}}. Color of the guidelines is determined by \code{plotColors\$guides}.

\item[\code{yLim}] numeric vector; manually set the limits of the plotting area (eg. \code{yLim=c(min,max)}). Used to format the y-axis by default but will modify the x-axis if \code{side=\LinkA{TRUE}{TRUE}}.

\item[\code{rotateLabels}] logical; sets \code{las=2} for the x-axis category labels. Will affect y-axis if \code{side=\LinkA{TRUE}{TRUE}}. Note that this may not work well if long names or with subgrouped data.

\item[\code{rotateY}] logical; sets \code{las=2} for the y-axis major tick-mark labels. Will affect x-axis if \code{side=\LinkA{TRUE}{TRUE}}.

\item[\code{theme}] list; A \code{NicePlots} plotColor list from a theme.

\item[\code{plotColors}] list; a named list of vectors of colors that set the color options for all NicePlot functions. Names left unspecified will be added and set to default values automatically.

\item[\code{trim}] positive numeric; passed to \code{threshold} argument of \code{\LinkA{quantileTrim}{quantileTrim}} if any data points are so extreme that they should be removed before plotting and downstream analysis. Set to \code{\LinkA{FALSE}{FALSE}} to disable.

\item[\code{logScale}] positive numeric; the base for the for log scale data transformation calculated as \code{log(x+1,logScale)}.

\item[\code{axisText}] character; a length two character vector containing text to be prepended or appended to the major tick labels, respectively.

\item[\code{minorGuides}] logical; draws guidelines at minor tick-marks

\item[\code{extendTicks}] logical; extends minor tick-marks past the first and last major tick to the edge of the graph provided there is enough room. Works for both log-scale and regular settings.

\item[\code{subGroup}] logical; use additional column in \code{by} to group the data within each level of the major factor.

\item[\code{expLabels}] logical; prints the major tick labels is \eqn{logScale^{x}}{} instead of the raw value

\item[\code{sidePlot}] logical; switches the axis to plot horizontally instead of vertically.

\item[\code{subGroupLabels}] character vector; sets the labels used for the \code{subGroup} factor. Defaults to the levels of the factor.

\item[\code{strictLimits}] logical; eliminates padding on the value axis so 0 can be flush with the x-axis. Defaults to \code{\LinkA{FALSE}{FALSE}}.

\item[\code{legend}] logical/character; Draw a legend in the plot margins. If a character string is given it will overide the factor name default for the legend title.

\item[\code{pointHighlights}] logical; Is pointHightlights turned on? This is used to determin with column of \code{by} should be used for legend factor levels.
\end{ldescription}
\end{Arguments}
%
\begin{Details}\relax
This function does all the hard work of setting up the x and y axis for plotting as well as optionally log transforming and/or trimming the data of outliers. In particular, it adds much more robust support for plotting of log transformed data and subgrouping of primary vectors. Other features include the addition of both major and minor guidelines, support for horizontal plotting and improved label formatting options.
\end{Details}
%
\begin{Value}
formats the plotting area and returns a named list with 'data' and 'labels' corresponding to the trimmed and/or transformed data and the labels for the primary factors, respectively.
\end{Value}
%
\begin{SeeAlso}\relax
\code{\LinkA{axisTicks}{axisTicks}}, \code{\LinkA{axis}{axis}}, \code{\LinkA{makeLogTicks}{makeLogTicks}}, \code{\LinkA{facetSpacing}{facetSpacing}}
\end{SeeAlso}
%
\begin{Examples}
\begin{ExampleCode}
todo<-1

\end{ExampleCode}
\end{Examples}
\inputencoding{utf8}
\HeaderA{prepNiceData}{Prepare and print basic statistics for niceBox and niceVio}{prepNiceData}
%
\begin{Description}\relax
Uses filtred data with subgroup and factor information to calculate quartile data for display and plotting.
\end{Description}
%
\begin{Usage}
\begin{verbatim}
prepNiceData(prepedData, by, subGroup = FALSE, outliers = TRUE, filter,
  groupNames, plotLoc, width = 1, flipFacts = FALSE, verbose = FALSE)
\end{verbatim}
\end{Usage}
%
\begin{Arguments}
\begin{ldescription}
\item[\code{prepedData}] list; a list object returned by \code{\LinkA{prepCategoryWindow}{prepCategoryWindow}}

\item[\code{by}] factor or dataframe of factors; One or more factors that control how the data is grouped. The first column is the primary grouping factor and the second and thrid columns are used for sub-grouping and highlighting as needed.

\item[\code{subGroup}] logical; Should the data be faceted into subgroups within the primary factor levels. Ignored if \code{by} is a \code{\LinkA{factor}{factor}}.

\item[\code{outliers}] positive numeric; number of interquartile ranges (IQR) past the Q1 (25\%) and Q3 (75\%) cumulative distribution values. Outliers are often defined as \eqn{1.5 \times IQR}{} and extreme outliers are more than \eqn{3 \times IQR}{} away from the inner 50\% data range.

\item[\code{filter}] logical vector; Used to further filter the data if necissary.

\item[\code{groupNames}] character; A character vector for the primary group names

\item[\code{plotLoc}] numeric vector; A vector indicating where each element should be plotted

\item[\code{width}] numeric; A multiplier that controls how wide the ploting elements will be. Setting \code{width=1.1} would result in plot elements being 10\% wider.

\item[\code{flipFacts}] logical; When a dataframe of values is given, column names are used as a secondary grouping factor by default. Setting \code{flipFacts=\LinkA{TRUE}{TRUE}} makes the column names the primary factor and \code{by} the secondary factor.

\item[\code{verbose}] logical; Will print summary statistics to the screen if set to \code{\LinkA{TRUE}{TRUE}}. The function will return the calculations either way.
\end{ldescription}
\end{Arguments}
%
\begin{Details}\relax
To aid in data interpretation and exploration, quartile distribution statistics are calculated for each group and subgroup
if specified. For \code{\LinkA{niceBox}{niceBox}} this data is also used to plot the data. The data is parsed by checking \code{outlier} and \code{subGroup} status
as weel as checking if either \code{prepedData} or \code{by} are a \code{\LinkA{data.frame}{data.frame}} or a \code{\LinkA{vector}{vector}}.
\end{Details}
%
\begin{SeeAlso}\relax
\code{\LinkA{niceBox}{niceBox}}, \code{\LinkA{niceVio}{niceVio}}, \code{\LinkA{niceDots}{niceDots}}
\end{SeeAlso}
%
\begin{Examples}
\begin{ExampleCode}
data(iris)
filter<-rep(TRUE,length(iris$Species))
loc<-seq(1,length(levels(iris$Species)))
data<-list(data=iris[,1:4])
myData<-prepNiceData(data,by=iris$Species,filter=filter,plotLoc=loc,
    groupNames=levels(iris$Species),outliers=FALSE)
print(myData)

\end{ExampleCode}
\end{Examples}
\inputencoding{utf8}
\HeaderA{quantileTrim}{Filter data by interquartile range}{quantileTrim}
%
\begin{Description}\relax
\code{quantileTrim} takes a numeric vector and removes data points that fall more than \code{threshold} * the interquartile range outside of the interquartile range. If \code{returnFilter} is set to TRUE then the function returns a named list with the trimmed data and a logical vector
\end{Description}
%
\begin{Usage}
\begin{verbatim}
quantileTrim(x, threshold = 3, na.rm = FALSE, returnFilter = FALSE)
\end{verbatim}
\end{Usage}
%
\begin{Arguments}
\begin{ldescription}
\item[\code{x}] a numeric vector or a object compatible with the \code{\LinkA{quantile}{quantile}} function

\item[\code{threshold}] numeric; the number of interquartile ranges out side of the inner 50\% range of the data to use as a cutoff from trimming. Typical values include 1.5 for outliers and 3 for extreme outliers.

\item[\code{na.rm}] logical; if true will remove all \code{\LinkA{NA}{NA}} values from \code{x} before analyzing the data.

\item[\code{returnFilter}] logical; will cause the function to return a list including with both the trimmed data and a logical vector that can be used to filter objects of the same length as \code{x}.
\end{ldescription}
\end{Arguments}
%
\begin{Details}\relax
The interquartile range (IQR) also known as the H-spread, represents the range encompassing the middle 50
This is is used to as a measure of dispersion around the median and more frequently to detect outlier data points.
Here data points are filtered if \eqn{x <  Q_{1} - threshold\times IQR}{} and \eqn{x > Q_{3} + threshold\times IQR}{} where \eqn{Q_{1}}{} and \eqn{Q_{3}}{} represent the cumulative 25
\end{Details}
%
\begin{Value}
The trimmed numeric vector or a \code{returnFilter} is \code{\LinkA{TRUE}{TRUE}} then a named list labeled data and filter is returned with the trimmed data and the logical filtering vector, respectively.
\end{Value}
%
\begin{SeeAlso}\relax
\code{\LinkA{quantile}{quantile}}.
\end{SeeAlso}
%
\begin{Examples}
\begin{ExampleCode}
x<-rnorm(1000)
paste0(mean(x)," (",range(x),")")
x<-quantileTrim(x,threshold=1.5)
paste0(mean(x)," (",range(x),")")

#Example using the filter function:
myData<-c(NA,rnorm(100),NA,NA,rnorm(100),NA,NA,NA,rnorm(300),NA,10000)
myIndex<-1:508
newData<-quantileTrim(myData,na.rm=TRUE,returnFilter=TRUE)
identical(newData$data,myData[newData$filter])
\end{ExampleCode}
\end{Examples}
\inputencoding{utf8}
\HeaderA{setAlpha}{add alpha transparency to a named color}{setAlpha}
%
\begin{Description}\relax
Takes a named color such as "red" or "darkgreen" and adds a level of transparancy based on the alpha setting.
\end{Description}
%
\begin{Usage}
\begin{verbatim}
setAlpha(x, alpha = 0.2)
\end{verbatim}
\end{Usage}
%
\begin{Arguments}
\begin{ldescription}
\item[\code{x}] character string; a text string corresponding to an R color

\item[\code{alpha}] numeric [0-1]; sets the level of transparency.
\end{ldescription}
\end{Arguments}
%
\begin{Details}\relax
\code{setAlpha} is a convenience function that uses the \code{\LinkA{col2rgb}{col2rgb}} and \code{\LinkA{rgb}{rgb}} to add transparancy to named colors.
\end{Details}
%
\begin{Value}
An rbg color with transparancy \code{alpha}.
\end{Value}
%
\begin{SeeAlso}\relax
\code{\LinkA{makeColorMatrix}{makeColorMatrix}}, \code{\LinkA{rainbow}{rainbow}}, \code{\LinkA{col2rgb}{col2rgb}}, \code{\LinkA{rgb}{rgb}}.
\end{SeeAlso}
%
\begin{Examples}
\begin{ExampleCode}
plot(1,1,col="white",xlim=c(0,10),ylim=c(0,10))
rect(1,1,7,7,col=setAlpha("darkblue"))
rect(3,3,9,9, col=setAlpha("red"))

\end{ExampleCode}
\end{Examples}
\inputencoding{utf8}
\HeaderA{vioplot}{violin plot}{vioplot}
%
\begin{Description}\relax
Produce violin plot(s) of the given (grouped) values.
\end{Description}
%
\begin{Usage}
\begin{verbatim}
vioplot(x, ..., range = 1.5, h = NULL, ylim = NULL, names = NULL,
  horizontal = FALSE, col = "magenta", border = "black", lty = 1,
  lwd = 1, rectCol = "black", colMed = "white", pchMed = 19, at,
  add = FALSE, wex = 1, drawRect = TRUE)
\end{verbatim}
\end{Usage}
%
\begin{Arguments}
\begin{ldescription}
\item[\code{x}] data vector

\item[\code{...}] additional data vectors

\item[\code{range}] a factor to calculate the upper/lower adjacent values

\item[\code{h}] the height for the density estimator, if omit as explained in sm.density, h will be set to an optimum

\item[\code{ylim}] y limits

\item[\code{names}] one label, or a vector of labels for the datas must match the number of datas given

\item[\code{horizontal}] logical. horizontal or vertical violins

\item[\code{col, }] color for the interior of the violin

\item[\code{border}] color border lines for the violin plot

\item[\code{lty}] numeric, R line type for the border lines in the violin plot

\item[\code{lwd}] numeric, line width for the border lines in the violin plot

\item[\code{rectCol, }] colMed, pchMed Graphical parameters to control the look of the box

\item[\code{colMed}] color of the median marker

\item[\code{pchMed}] shape of the median marker using the \code{pch} R graphical parameter

\item[\code{at}] position of each violin. Default to \code{1:n}

\item[\code{add}] logical. if FALSE (default) a new plot is created

\item[\code{wex}] relative expansion of the violin.

\item[\code{drawRect}] logical. the box is drawn if \code{TRUE}.
\end{ldescription}
\end{Arguments}
%
\begin{Details}\relax
This is a modified version of the \code{vioplot} package from CRAN.
Code and documentation by Daniel Adler and repurposed here with the author's permission.
Original code obtained from \code{https://github.com/cran/vioplot/}.

A violin plot is a combination of a box plot and a kernel density plot.
Specifically, it starts with a box plot. It then adds a rotated kernel density plot to each side of the box plot.
\end{Details}
%
\begin{Author}\relax
Daniel Adler \email{dadler@uni-goettingen.de}
\end{Author}
%
\begin{References}\relax
Hintze, J. L. and R. D. Nelson (1998).  \emph{Violin plots: a box plot-density trace synergism.}  The American Statistician, 52(2):181-4.
\end{References}
%
\begin{SeeAlso}\relax
\code{\LinkA{boxplot}{boxplot}} \code{\LinkA{sm}{sm}}
\end{SeeAlso}
%
\begin{Examples}
\begin{ExampleCode}
# box- vs violin-plot
par(mfrow=c(2,1))
mu<-2
si<-0.6
bimodal<-c(rnorm(1000,-mu,si),rnorm(1000,mu,si))
uniform<-runif(2000,-4,4)
normal<-rnorm(2000,0,3)
vioplot(bimodal,uniform,normal)
boxplot(bimodal,uniform,normal)
# add to an existing plot
x <- rnorm(100)
y <- rnorm(100)
plot(x, y, xlim=c(-5,5), ylim=c(-5,5))
vioplot(x, col="tomato", horizontal=TRUE, at=-4, add=TRUE,lty=2, rectCol="gray")
vioplot(y, col="cyan", horizontal=FALSE, at=-4, add=TRUE,lty=2)
\end{ExampleCode}
\end{Examples}
\printindex{}
\end{document}
